\section{Inngangur}
Lestin Rocet! \cite{mcgowan2010rail}

Markmiðið er að hanna og framleiða lest sem er stírð af tölvu sem er ekki á lestini
það gæti þarfnast töluverða vírs forritun til að gera þennan "draum" að veruleika.
Við töldum okkur ekki ætlast að fara neit djúft í rafmags fræðina en hún kom til okkar.
Við völdum þetta verk út frá ASDF movie "I like trains" og af því að Elon Musk er 
fyrirmyndin okkar fyrir skrítna og sharmandi vinnuhætti sem allir geta nutið.
Við áttuðum okkur seinna á því að við þurftum að forrita með vírnum frekar en tölvuni
því að allt transmission til tölvulausu lestarinnar er í gegnum ein vír(tein) og 
ein vír(tein) til að svara um skilavoðin sem fer beint í tölvuna,
3 input og 1 output sem samsvarar áhveðnum skilirðum.
Inputin geta verið eitt af þrem skilaboðum sem er í raun bara áhveðin stór spenna
eða ekki spenna, og outputið táknar alltaf bara hvort lestin sé á teinunum.
Lestin er partur af closed circuit, um leið og lestin fer af teinunum verður 
allt kerfið open circuit og er ekki lengur functional.
Í hvert skifti sem lestin á að vera á ferð, er talvan að lesa á response'ið og 
segja til um hvort allt er að virka eins og það á að gera samkvæmt áætlun.
Hægt er að keyra lestina í ákveðin tíma eða (eins og við ætlum okkur) að keyra 
þangað til að circuit'ið er rofið með (til dæmis) teipi eða öðrum plast einangrunum,
með þessum hætti getum við sagt til um hvar lestin er þegar hún stríkur yfir, 
aukalega væri hægt (með töluverði forritun) að senda skilaboð með rofningum
eins og 2 rofningar í röð innan við 2 sekondur  til að merkja við stöð númer 2,
eða ef nákvæmnin leifir það binary skilaboð eins og teip, ekki teip, teip, teip
væri túlkað sem 1011 bin eða 11 desimal. með þessum hætti væri hægt að senda margvísleg
og flókin skilaboð frá lestini til að segja til um hvað talvan á að gera.
Við áttuðum okkur ekki á því hversu mikla vinnu við myndum þurfa að setja í vírin til
að triggja það að allt rennur vel og smoothly, eingin of hitnun og svoleiðis.
